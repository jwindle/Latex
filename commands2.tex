% Regarding mathematical font styles we have the following:
%    * \textstyle - default in the running text and in array environment
%    * \displaystyle - default for displayed equations
%    * \scriptstyle - default for first-level sub and superscripts
%    * \scriptscriptstyle - default for higher-level sub and superscripts

% Page Layout
%\setlength{\oddsidemargin}{0.0in}
%\setlength{\textwidth}{6.5in}
%\setlength{\textheight}{8in}
%\parindent 0in
%\parskip 12pt

% Set Fancy Style
%\pagestyle{fancy}

% Mathematical Organization
% Associate Theorem counter with section counter
%\theoremstyle{definition}
%\newtheorem{theorem}{Theorem:}[section]
%\newtheorem{lemma}[theorem]{Lemma:}
%\newtheorem{proposition}[theorem]{Proposition:}
%\newtheorem{corollary}[theorem]{Corollary:}
%\newtheorem{definition}[theorem]{Definition:}
%\newtheorem{remark}[theorem]{Remark:}
%\newtheorem{example}[theorem]{Example:}
%\newtheorem{claim}[theorem]{Claim:}

% Create a new theorem counter.
\ifdefined \theorem
\else
\theoremstyle{definition}
\newtheorem{theorem}{Theorem}
\newtheorem{lemma}[theorem]{Lemma}
\newtheorem{proposition}[theorem]{Proposition}
\newtheorem{corollary}[theorem]{Corollary}
\newtheorem{definition}[theorem]{Definition}
\newtheorem{remark}[theorem]{Remark}
\newtheorem{example}[theorem]{Example}
\newtheorem{claim}[theorem]{Claim}
\newtheorem{fact}[theorem]{Fact}
\newtheorem{aside}[theorem]{Aside}
\newtheorem{exercise}[theorem]{Exercise}
\newtheorem{conjecture}[theorem]{Conjecture}

% Creating a new theorem generates a
% counter of the same name.
\newtheorem{problem}{Problem}
% If I want to change the appearance I can
% change the \the<countername> function, i.e.
% \renewcommand{\theproblem}{\arabic{problem}}
% I also can just write the counter directly with
% \arabic{problem}, \alph{problem}, etc.
\fi

% I couldn't get this to work:
% \newtheoremstyle{stylename} % name of the style to be used
%  {spaceabove} measure of space to leave above the theorem. E.g.: 3pt
%  {spacebelow} measure of space to leave below the theorem. E.g.: 3pt
%  {bodyfont}% name of font to use in the body of the theorem
%  {indent}% measure of space to indent
%  {headfont}% name of head font
%  {headpunctuation}% punctuation between head and body
%  {headspace}% space after theorem head
%  {headspec}% Manually specify head

% Convex Analysis Commands
\newcommand{\core}{\text{core }}
\newcommand{\dom}{\text{dom }}
% \newcommand{\ip}[2]{\langle #1, #2 \rangle}
\newcommand{\ip}[1]{\langle #1 \rangle}
\newcommand{\epi}{\text{epi }}
\newcommand{\cl}{\text{cl }}
\newcommand{\newpar}{\vspace{12pt} \noindent}
\newcommand{\mE}{\mathbb{E}}
\newcommand{\mycomment}[1]{}
\newcommand{\expct}{\mathbb{E}}
\newcommand{\argmax}[1]{\underset{#1}{\operatorname{argmax}}\;}
\newcommand{\argmin}[1]{\underset{#1}{\operatorname{argmin}}\;}

% Text Manipulation
\newcommand{\myblue}[1]{\textcolor{blue}{#1}}
\newcommand{\myred}[1]{\textcolor{red}{#1}}
\newcommand{\mygreen}[1]{\textcolor{green}{#1}}
\newcommand{\mymagenta}[1]{\textcolor{magenta}{#1}}
\newcommand{\bsym}[1]{\boldsymbol{#1}}

% Probability Commands
\newcommand{\id}{\text{d}}
\newcommand{\sd}{\circ \text{d}}
\newcommand{\FV}{\text{FV}}
% \newcommand{\crossvar}[2]{ \langle #1, #2 \rangle}
% \newcommand{\quadvar}[1]{\langle #1 \rangle}
% Just use \ip{X} or \ip{X, Y}.
\newcommand{\chose}[2]{\Big( \begin{array}{c} #1 \\ #2 \end{array} \Big) }
\newcommand{\E}{\text{E}}
\newcommand{\convprob}{\stackrel{p}{\longrightarrow}}
\newcommand{\convlaw}{\stackrel{\mcL}{\longrightarrow}}
\newcommand{\sigalg}{$\sigma$-algebra }
\newcommand{\RA}{\Rightarrow}
\newcommand{\icatorset}[1]{ \1_{\{ #1 \}} }
\newcommand{\Cov}{\text{Cov}}
\newcommand{\Var}{\text{Var}}
\newcommand{\Cor}{\text{Cor}}
\newcommand{\cov}{\text{cov}}
\newcommand{\var}{\text{var}}
\newcommand{\cor}{\text{cor}}
\newcommand{\sdv}{\text{sd}}
\newcommand{\vect}{\text{vec}}
\newcommand{\vech}{\text{vech}}
\newcommand{\etr}{\text{etr}}
\newcommand{\col}{\text{col}}
\newcommand{\one}{\bold{1}}

%Legacy command
\newcommand{\cF}{\mathcal{F}}

% Analysis Commands
\newcommand{\weaklim}{\buildrel{w}\over{\rightarrow}}
\newcommand{\distlim}{\buildrel{D}\over{\rightarrow}}
\newcommand{\ep}{\varepsilon}
\newcommand{\dvg}{\text{ div }}
\newcommand{\tr}{\text{ tr }}
\newcommand{\tripnorm}[1]{|||#1|||}

% Blackboard
\newcommand{\bbA}{\mathbb{A}}
\newcommand{\bbB}{\mathbb{B}}
\newcommand{\bbC}{\mathbb{C}}
\newcommand{\bbD}{\mathbb{D}}
\newcommand{\bbE}{\mathbb{E}}
\newcommand{\bbF}{\mathbb{F}}
\newcommand{\bbG}{\mathbb{G}}
\newcommand{\bbH}{\mathbb{H}}
\newcommand{\bbI}{\mathbb{I}}
\newcommand{\bbJ}{\mathbb{J}}
\newcommand{\bbK}{\mathbb{K}}
\newcommand{\bbL}{\mathbb{L}}
\newcommand{\bbM}{\mathbb{M}}
\newcommand{\bbN}{\mathbb{N}}
\newcommand{\bbO}{\mathbb{O}}
\newcommand{\bbP}{\mathbb{P}}
\newcommand{\bbQ}{\mathbb{Q}}
\newcommand{\bbR}{\mathbb{R}}
\newcommand{\bbS}{\mathbb{S}}
\newcommand{\bbT}{\mathbb{T}}
\newcommand{\bbU}{\mathbb{U}}
\newcommand{\bbV}{\mathbb{V}}
\newcommand{\bbW}{\mathbb{W}}
\newcommand{\bbX}{\mathbb{X}}
\newcommand{\bbY}{\mathbb{Y}}
\newcommand{\bbZ}{\mathbb{Z}}
% Specialized Blackboard
\newcommand{\R}{\mathbb{R}}
%\newcommand{\1}{\mathbbm{1}}
%\newcommand{\1}{\mathbf{1}}

% match caligraphy
\newcommand{\mcA}{\mathcal{A}}
\newcommand{\mcB}{\mathcal{B}}
\newcommand{\mcC}{\mathcal{C}}
\newcommand{\mcD}{\mathcal{D}}
\newcommand{\mcE}{\mathcal{E}}
\newcommand{\mcF}{\mathcal{F}}
\newcommand{\mcG}{\mathcal{G}}
\newcommand{\mcH}{\mathcal{H}}
\newcommand{\mcI}{\mathcal{I}}
\newcommand{\mcJ}{\mathcal{J}}
\newcommand{\mcK}{\mathcal{K}}
\newcommand{\mcL}{\mathcal{L}}
\newcommand{\mcM}{\mathcal{M}}
\newcommand{\mcN}{\mathcal{N}}
\newcommand{\mcO}{\mathcal{O}}
\newcommand{\mcP}{\mathcal{P}}
\newcommand{\mcQ}{\mathcal{Q}}
\newcommand{\mcR}{\mathcal{R}}
\newcommand{\mcS}{\mathcal{S}}
\newcommand{\mcT}{\mathcal{T}}
\newcommand{\mcU}{\mathcal{U}}
\newcommand{\mcV}{\mathcal{V}}
\newcommand{\mcW}{\mathcal{W}}
\newcommand{\mcX}{\mathcal{X}}
\newcommand{\mcY}{\mathcal{Y}}
\newcommand{\mcZ}{\mathcal{Z}}
% specialized caligraphy commands
\newcommand{\B}{\mathcal{B}}

% to make bold easy
\newcommand{\mb}[1]{\mathbf{#1}}

% Various Boxes
\newcommand{\boxedeqn}[1]{%
  \[\fbox{%
      \addtolength{\linewidth}{-2\fboxsep}%
      \addtolength{\linewidth}{-2\fboxrule}%
      \begin{minipage}{\linewidth}%
      \begin{equation}#1\end{equation}%
      \end{minipage}%
    }\]%
}

\newcommand{\boxeddm}[1]{%
  \[\fbox{%
      \addtolength{\linewidth}{-2\fboxsep}%
      %\addtolength{\linperpendicularewidth}{-2\fboxrule}%
      \addtolength{\linewidth}{-2\fboxrule}%
      \begin{minipage}{\linewidth}%
      \begin{displaymath}#1\end{displaymath}%
      \end{minipage}%
    }\]%
}

\newcommand{\boxit}[1]{
  \[\fbox{%
      \addtolength{\linewidth}{-2\fboxsep}%
      \addtolength{\linewidth}{-2\fboxrule}%
      \begin{minipage}{\linewidth}%
      #1
      \end{minipage}%
    }\]%
}

% Esoteric Commands
\newcommand{\XT}{\tilde{X}}
\newcommand{\tdx}[1]{{#1}_x}

\newcommand{\tm}{\text}
\newcommand{\half}{\frac{1}{2}}
\newcommand{\ra}{\rightarrow}
\newcommand{\del}{\partial}
\newcommand{\diag}{\text{diag}}

% Fincance commands
\newcommand{\dol}{\$ }
\newcommand{\deu}{DEU}

% Text commands
\newcommand{\ttt}[1]{\texttt{#1}}
